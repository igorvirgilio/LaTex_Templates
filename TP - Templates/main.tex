\documentclass{llncs}
% Command to remove the content counter
\setcounter{secnumdepth}{0}

\usepackage[utf8]{inputenc}
\usepackage[T1]{fontenc}
\usepackage[brazilian]{babel}
\usepackage{csquotes}
\usepackage{hyphenat}
\hyphenation{pro-ble-ma}
\usepackage{imakeidx}
\makeindex[columns=3, title=Alphabetical Index, intoc]
\usepackage{enumerate}
\usepackage{amssymb}
\usepackage{amsmath}
\usepackage{array}

% Better handle of numeric things.
% Configured to French style (same 
% as Brazilian conventions)
\usepackage[locale=FR]{siunitx}
\usepackage{hyperref}
\usepackage{graphicx}
\graphicspath{ {images/} }
\usepackage{longtable}
\usepackage{listings}
\setcounter{tocdepth}{1}
\usepackage{tocbibind}

\begin{document}
\begin{titlepage}           
\end{titlepage}
%%%%%%%%%%%%%%%%%%% TITLE PAGE
\begin{titlepage}
  \begin{center}
    \vspace*{1cm}
    %%% TITLE
    \Huge
    \textbf{Tarefa Prática 2}
    \vspace{0.5cm}
    
    %%% CURRICULAR UNIT
    \LARGE
    Modelação e Caracterização de Tráfego
    %%% AUTHORS    
    \vspace{1.0cm}
    \small
    \textbf{\\PG39254 - Igor Araújo\\PG39255 - Matheus Gonçalves\\PG41017 - I-Ping}
    
    %%% LOGO
    \vspace{1.0cm}
    \begin{figure}[ht]
    \includegraphics[width=0.8\textwidth]{uminho.jpg}
    \centering
    \end{figure}
    
    % FOOTER
    \vspace{4.5cm}
    Departamento de Informática\\
    Universidade do Minho\\
    Braga - Portugal\\
    \today
          
  \end{center}
\end{titlepage}

\tableofcontents

\clearpage

\section{Objetivo}

%
O objetivo desse trabalho é realizar a captura, visualização, análise e filtragem de tráfego de rede, onde 
no final desse relatório o grupo vai estar mais familiarizado com as ferramentas e os conceitos de captura e análise de tráfego. 

%
\section{Parte 1 - Captura e análise de tráfego}

%\subsection{Primeira Análise}

\begin{enumerate}[a]
  \item Inicie a captura de tráfego na interface de rede disponível. Faça uma primeira análise comparativa dos
  cabeçalhos e formatos dos PDUs do protocolos TCP, UDP e IP. Identifique para cada um deles os campos
  geralmente utilizados na classificação de tráfego.
  % Start new paragraph using alignment Justification 
  \begin{flushleft}
    INICIO RESPOSTA
  \end{flushleft}
\end{enumerate}

\begin{enumerate}[b]
  \item Utilizando o sniffer em modo de captura, proceda à invocação de várias aplicações conhecidas, nomeadamente:\\*
  \begin{itemize}
      \item Acesso via browser ao URL: http://marco.uminho.pt
      \item Acesso ftp (anonymous): ftp.di.uminho.pt
      \item Acesso em tftp para router-ext (193.136.9.33)
      \item Acesso via telnet para router-ext (193.136.9.33) ou para\\*
      router-lab (192.168.90.254)
      \item Acesso ssh para qualquer host da sala de aula
      \item Resolução de nomes usando nslookup www.uminho.pt
      \item traceroute cisco.uminho.pt\\*
  \end{itemize}
  \par e construa uma tabela onde, para cada aplicação, conste o protocolo de transporte e a porta de atendimento do
  servidor (quando aplicável).
  
  % Start new paragraph using alignment Justification 
  \begin{flushleft}
    INICIO RESPOSTA
  \end{flushleft}
\end{enumerate}





\begin{figure}[ht]
\includegraphics[scale=0.5]{uminho.jpg}
\centering
\caption{\enquote{X} em modelos de tamanhos diferentes}
\label{fig:xs}
\end{figure}

%
\section{Resultados}
%
Os resultados dos experimentos se encontram na tabela 1. 

\begin{table}
\centering
\begin{tabular}{|c|S|c|c|c|c|c|c|c|c|}
\hline
\multicolumn{3}{|c|}{Parâmetros} & \multicolumn{5}{|c|}{Geração até chegar à solução} & \multicolumn{2}{|c|}{Desempenho} \\ 
\hline
Grade & Mutação & População & \multicolumn{2}{|c|}{95\% de confiança} & 1º Quartil & Mediana & 3º Quartil & Indivíduos & IC \\ 
\hline
3x3 & 0.1 & 10 & 2 & 1000 & 16 & 167 & 435 & 1670 & 96,17\%\\ 
\hline
3x3 & 0.01 & 10 & 1 & 1000 & 12 & 132 & 383 & 1320 & 92,42\%\\ 
\hline
3x3 & 0.001 & 10 & 2 & 1000 & 40 & 197 & 430 & 1970 & 97,87\%\\ 
\hline
3x3 & 0 & 10 & 2 & 1000 & 32 & 135 & 404 & 1350 & 92,85\%\\ 
\hline
3x3 & 0.1 & 100 & 1 & 6 & 2 & 3 & 4 & 300 & 44,37\%\\ 
\hline
3x3 & 0.01 & 100 & 1 & 7 & 2 & 3 & 4 & 300 & 44,37\%\\ 
\hline
3x3 & 0.001 & 100 & 1 & 7 & 2 & 3 & 4 & 300 & 44,37\%\\ 
\hline
3x3 & 0 & 100 & 1 & 8 & 2 & 3 & 4 & 300 & 44,37\%\\ 
\hline
3x3 & 0.1 & 1000 & 1 & 2 & 1 & 1 & 1 & 1000 & 85,84\%\\ 
\hline
3x3 & 0.01 & 1000 & 1 & 2 & 1 & 1 & 1 & 1000 & 85,84\%\\ 
\hline
3x3 & 0.001 & 1000 & 1 & 2 & 1 & 1 & 1 & 1000 & 85,84\%\\ 
\hline
3x3 & 0 & 1000 & 1 & 3 & 1 & 1 & 1 & 1000 & 85,84\%\\ 
\hline
4x4 & 0.1 & 10 & 406 & 1000 & 1000 & 1000 & 1000 & 10000 & \\ 
\hline
4x4 & 0.01 & 10 & 535 & 1000 & 1000 & 1000 & 1000 & 10000 & \\ 
\hline
4x4 & 0.001 & 10 & 241 & 1000 & 1000 & 1000 & 1000 & 10000 & \\ 
\hline
4x4 & 0 & 10 & 185 & 1000 & 1000 & 1000 & 1000 & 10000 & \\ 
\hline
4x4 & 0.1 & 100 & 5 & 27 & 11 & 14 & 17 & 1400 & 2,11\%\\ 
\hline

\hline
\hline\end{tabular}
\label{tab:resultados}
\caption{\small{Resultados brutos}} 
\end{table}

\newpage
\section{Anexo I}
\label{sec:anexo_I}

\begin{table}
\centering
\begin{longtable}{p{.20\textwidth} p{.20\textwidth} p{.20\textwidth} p{.40\textwidth}}
\label{tbl:2}

%\begin{tabular}{p{2cm}|p{2cm}|p{2cm}|p{8cm}}

\\\textbf{Test} & \textbf{Metric} & \textbf{Plataform} & \textbf{Description}\\
\hline\hline
Download (TCP) & Download speed & Whiteboxes, Routers, Android, iOS & The download speed in Mbps when downloading (using TCP) random bytes from a test server \\ 
\hline 
  & TCP Retransmissions & Whiteboxes, Routers & The number of retransmitted TCP segments/packets \\ 
\hline
  & Burst download speed & Whiteboxes, Routers & The download speed during the first 5 seconds of a test \\ 
\hline 
  & Sustained download speed & Whiteboxes, Routers & The download speed of the test during the last 5 seconds \\ 
\hline 
  & Percentage of Best & Whiteboxes, Routers & Download speed result as a percentage of the user's best ever result \\ 
\hline 
  & Percentage of Advertised & Whiteboxes, Routers & Download speed result as a percentage of their package's advertised downstream speed \\ 
\hline 
Download (HTML5) & Download speed & Web & The download speed in Mbps when downloading (using TCP) random bytes from a test server using HTML5 APIs(WebSockets and Fetch) \\ 
\hline 
Download (Lightweight UDP) & Download speed & Whiteboxes, Routers & The download speed in Mbps when downloading (using UDP) from a test server, using less data than the TCP test \\ 
\hline 
Download (Hardware acceleratedUDP) & Download speed & Broadcom-based Routers & The download speed in Mbps when downloading (using UDP) random bytes from a test server \\
\hline 
\caption{Tabela com alguns exemplos}
\end{longtable}
\end{table}
%
\section{Conclusões}
%
O experimento é pequeno para mostrar dados conclusivos, mas mostra indícios do comportamento dos parâmetros de maneira bastante consistente. Obviamente, alterações no algoritmo ou alterações na forma como o problema é representado devem alterar esse comportamento.

Futuros trabalhos podem ser feitos usando-se uma metodologia parecida, mas com modificações no algoritmo e no problema para validar os dados aqui obtidos.

%
% ---- Bibliography ----
%
\begin{thebibliography}{5}
%

\bibitem {castro}
de Castro, L.N.: Fundamentals of Natural Computing: Basic Concepts, Algorithms, and Applications. CRC Press (2006).

\bibitem{racket}
Felleisen, M., Findler, R.B., Flatt, M.: The Racket Manifesto. LIPIcs-Leibniz. (2015).

\bibitem{deb:agra}
Deb, K., Agrawal, S.: Understanding interactions among genetic algorithm parameters. Foundations of Genetic Algorithms. (1999).

\end{thebibliography}

\end{document}
