\documentclass{llncs}
% Command to remove the content counter
\setcounter{secnumdepth}{0}

\usepackage[utf8]{inputenc}
\usepackage[T1]{fontenc}
\usepackage[brazilian]{babel}
\usepackage{csquotes}
\usepackage{hyphenat}
\hyphenation{pro-ble-ma}
\usepackage{imakeidx}
\makeindex[columns=3, title=Alphabetical Index, intoc]
\usepackage{enumerate}
\usepackage{amssymb}
\usepackage{amsmath}
\usepackage{array}

% Better handle of numeric things.
% Configured to French style (same 
% as Brazilian conventions)
\usepackage[locale=FR]{siunitx}
\usepackage{hyperref}
\usepackage{graphicx}
\graphicspath{ {images/} }
\usepackage{longtable}
\usepackage{listings}
\setcounter{tocdepth}{1}
\usepackage{tocbibind}

\begin{document}
\begin{titlepage}           
\end{titlepage}
%%%%%%%%%%%%%%%%%%% TITLE PAGE
\begin{titlepage}
  \begin{center}
    \vspace*{1cm}
    %%% TITLE
    \Huge
    \textbf{Tarefa Prática 2}
    \vspace{0.5cm}
    
    %%% CURRICULAR UNIT
    \LARGE
    Modelação e Caracterização de Tráfego
    %%% AUTHORS    
    \vspace{1.0cm}
    \small
    \textbf{\\PG39254 - Igor Araújo\\PG39255 - Matheus Gonçalves\\PG41017 - I-Ping}
    
    %%% LOGO
    \vspace{1.0cm}
    \begin{figure}[ht]
    \includegraphics[width=0.8\textwidth]{uminho.jpg}
    \centering
    \end{figure}
    
    % FOOTER
    \vspace{4.5cm}
    Departamento de Informática\\
    Universidade do Minho\\
    Braga - Portugal\\
    \today
          
  \end{center}
\end{titlepage}

\tableofcontents

\clearpage

\section{Objetivo}

%
O objetivo desse trabalho é realizar a captura, visualização, análise e filtragem de tráfego de rede, onde 
no final desse relatório o grupo vai estar mais familiarizado com as ferramentas e os conceitos de captura e análise de tráfego. 

%
\section{Parte I - Captura e análise de tráfego}

%\subsection{Primeira Análise}

\begin{enumerate}[\textbf{a)}]
  \item \textbf{ Inicie a captura de tráfego na interface de rede disponível. Faça uma primeira análise comparativa dos cabeçalhos e formatos dos PDUs do protocolos TCP, UDP e IP. Identifique para cada um deles os campos geralmente utilizados na classificação de tráfego:}
  \vspace{5mm}
  % Start new paragraph using alignment Justification 
    \par O protocolo TCP possui header que contém diversos campos, mas os campos que são utilizados geralmente para identificação e classificação de um tráfego são as portas de origem e destino, e da mesma maneira para o UDP. Além destas, para a melhor classificação do tráfego é utilizado também os campos da PDU da camada de redes IP, que utilizam os endereços de origem e destino IP o número de protocolo, assim é formada a 5 tupla. Em posse desses parâmetros é possível em muitos casos classificar o tráfego. Porém a cada dia novas aplicações com diversos tipos de tráfegos são enviadas através de tráfegos encriptados o que torna ainda mais difícil sua identificação e classificação. Pode-se observar tais campos mencionados na figura~\ref{fig:PDU}.
    
  
  \begin{figure}[h]
    \includegraphics[scale=0.65]{PDU.png}
    \centering
    \caption{Exemplificação de PDU.}
    \label{fig:PDU}
    \end{figure}
\end{enumerate}


\begin{enumerate}[\textbf{b)}]
  \item \textbf{Utilizando o sniffer em modo de captura, proceda à invocação de várias aplicações conhecidas, nomeadamente:}
  \vspace{5mm}
  \begin{itemize}
      \item Acesso via browser ao URL: http://marco.uminho.pt
      \item Acesso ftp (anonymous): ftp.di.uminho.pt
      \item Acesso em tftp para router-ext (193.136.9.33)
      \item Acesso via telnet para router-ext (193.136.9.33) ou para\\*
      router-lab (192.168.90.254)
      \item Acesso ssh para qualquer host da sala de aula
      \item Resolução de nomes usando nslookup www.uminho.pt
      \item traceroute cisco.uminho.pt
  \end{itemize}
  \par \textbf{e construa uma tabela onde, para cada aplicação, conste o protocolo de transporte e a porta de atendimento do
  servidor (quando aplicável).}
  
  % Start new paragraph using alignment Justification 
  \vspace{5mm}
   
  \begin{table}[h!]
    \centering
    \begin{tabular}{p{4.4cm}  p{3cm}  p{3cm}} 
     \hline
     \textbf{Protocolo de Transporte} & \textbf{Porta de Origem} & \textbf{Porta de Destino}\\ [1ex] 
     \hline\hline
     HTTP & 6 & 87837 \\ [1ex]
     FTP & 7 & 78 \\ [1ex]
     TFTP &  & 69 \\ [1ex]
     TELNET & 545 & 23 \\ [1ex]
     SSH & 88 & 22 \\ [1ex]
     DNS & 88 & 53 \\ [1ex]
     ICMP & 88 & 53 \\ [1ex] 
     \hline
    \end{tabular}
    \caption{Tabela de aplicações}
    \label{table:1}
    \end{table}


\end{enumerate}


\section{Parte 2 - Filtragem de tráfego}

\begin{enumerate}[\textbf{a)}]
  \item \textbf{Explore e descreva:}
  \begin{enumerate}[i]
    \item A utilidade dos filtros de captura e visualização;    
    \item A sintaxe e semântica dos filtros.
  \end{enumerate}
  \par\textbf{Dê alguns exemplos simples de utilização dos mesmos.}
  \begin{flushleft}
    INICIO RESPOSTA
  \end{flushleft}
\end{enumerate}


\begin{enumerate}[\textbf{b)}]
  \item  \textbf{Baseando-se nas tramas capturadas acima (1.b), e em outros exemplos que achar conveniente, explore a
  utilidade e utilização dos filtros de captura e visualização, nomeadamente na captura/visualização de:}
  \begin{itemize}
    \item protocolos aplicacionais;
    \item protocolos de transporte;
    \item endereços IP;
    \item pacotes com valores específicos nos campos principais dos cabeçalhos de transporte e rede (ver opção
    "+Expression");
    \item pacotes com flags de iniciação e termino de conexões TCP;
  \end{itemize}
  \par \textbf{Exemplifique a exploração que realizou, indicando a sintaxe utilizada nos filtros e, muito sucintamente os
  resultados obtidos.}
  \begin{flushleft}
    INICIO RESPOSTA
  \end{flushleft}
\end{enumerate}


\begin{enumerate}[\textbf{c)}]
  \item \textbf{Para uma das aplicações que usam o protocolo TCP (e.g. Telnet router-ext), explore a opção "Analyse - Follow TCP Stream". Indique os filtros automaticamente aplicados por essa opção. Discuta eventuais fragilidades
  de segurança e confidencialidade dos dados.}
  
  \vspace{5mm}
  \begin{flushleft}
    \par Com a opção de filtragem via menu Analyse > Follow > TCP Stream, é possível selecionar um pacote entra vários capturados e reunir todos os pacotes que pertencem ao mesmo stream de dados. Neste caso foi feito inicialmente uma filtragem pelo protocolo Telnet, conforme visto na figura~\ref{fig:stream1} abaixo:
    \begin{figure}[h]
      \includegraphics[scale=0.65]{stream1.png}
      \centering
      \caption{Exemplo lista de stream.}
      \label{fig:stream1}
    \end{figure}
  \end{flushleft}

  \begin{flushleft}
    \par Em seguida foi utilizado o menu Analyse > Follow > TCP Stream, mencionado anteriormente e com isso foi possível agrupar todos os pacotes pertencentes ao stream de pacotes pertencentes ao stream do pacote selecionado, inclusive aqueles que não são exclusivamente de protocolo Telnet, conforme visto a seguir na figura~\ref{fig:stream2}.
    \begin{figure}[h]
      \includegraphics[scale=0.65]{stream2.png}
      \centering
      \caption{Exemplo lista de stream.}
      \label{fig:stream2}
    \end{figure}
  \end{flushleft}

  \begin{flushleft}
    \par Como pode também ser visto na figura~\ref{fig:stream2} o filtro que é gerado pelo menu executado basicamente é filtrar na captura pelo stream TCP de número 6 que sintaticamente possui a expressão (tcp.stream eq 6), o trecho tcp.stream indica intuitivamente que quer se filtrar por streams TCP e a parte (eq 6) indica que o stream especifico que se deseja é o de número igual (eq) a 6.
    \par E o mais interessante do resultado da ação executado pelo menu selecionado é a reconstrução e apresentação das trocas de mensagens trocadas entre origem e destino, de tal forma que seja possível capturar e entender uma troca de mensagens por completo, se forem enviados em texto claro, que é o caso do protocolo Telnet. Tal resultado é visualizado na figura~\ref{fig:followstream}
    \begin{figure}[h]
      \includegraphics[scale=0.65]{followstream.png}
      \centering
      \caption{Montando o stream.}
      \label{fig:followstream}
    \end{figure}
  \end{flushleft}

  \begin{flushleft}
    \par Os trechos apresentados marcados em azul foram recebidos pelo destinatário e em vermelho pela a origem, que está a tentar aceder ao equipamento via protocolo Telnet. Assim vemos de forma clara o password que foi digitado pelo utilizador. Desta forma mostra a fragilidade do protocolo Telnet, bem como outros protocolos que transmitem suas mensagens via texto claro, caso sejam transmitidos conteúdos sensíveis como senhas, informações bancárias e outros, tais dados estarão expostos e a comprometer a confidencialidade das informações caso haja um utilizador malicioso a sniffar os pacotes que são transmitidos pela rede.
  \end{flushleft}
\end{enumerate}


\begin{enumerate}[\textbf{d)}]
  \item \textbf{Analise e identifique dados estatísticos da sua captura de pacotes.}

  \begin{flushleft}
    Dentre as capturas realizadas selecionou-se a referente ainda ao Telnet. Na tela principal já é possivel verificar a quantidade de pacotes capturados no total e quantos estão sendo exibidos, quando há um filtro aplicado.
    \par CRIAR TABELA AQUI:
    \par Quantidade de pacotes capturados: 352
    \par Total de pacotes exibidos: 50(14.2\%)
    \par Outra opção para se obter mais estatísticas é utilizar o menu Statistics, nele há uma lista de opções. Uma delas que é interessante é o Conversation, nesta são compiladas todas as conversas entre origm X e destindo Y para os protocolos Ethernet, Ipv4, Ipv6, TCP e UDP, sendo essas opções distribuídas em abas, conforme visto abaixo na figura~\ref{fig:conversation01}.
    \begin{figure}[h]
      \includegraphics[scale=0.65]{conversation01.png}
      \centering
      \caption{Tabela conversation.}
      \label{fig:conversation01}
    \end{figure}
  \end{flushleft}
\end{enumerate}
\begin{flushleft}
  \par Outra estatistica interessante é listagem hierárquica dos protocolos, nela pode-se ver a representatividade de cada protocolo e subprotocolo no total da captura. Essa estatística pode ser visualizada na figura~\ref{fig:statistic}.
  \begin{figure}[h]
    \includegraphics[scale=0.65]{statistic.png}
    \centering
    \caption{Tabela de estatística hierárquica.}
    \label{fig:statistic}
  \end{figure}
\end{flushleft}

\begin{flushleft}
  \par E outra forma de visualizar estatísticas é na opção File Properties do menu Statistics, que pode ser visualizado na figura~\ref{fig:details}
  \begin{figure}[h]
    \includegraphics[scale=0.65]{details.png}
    \centering
    \caption{File Properties.}
    \label{fig:details}
  \end{figure}
  \par Quantidade de pacotes capturados: 352
  \par Total de pacotes exibidos: 50(14.2\%)
  
\end{flushleft}

\section{Conclusão}
  \begin{flushleft}
    Ao longo desse trabalho, tivemos a oportunidade de desenvolver nossas habilidades na análise do tráfego gerado.
    Com fácil aprendizado, a ferramenta Wireshark se mostra extremamente poderosa e eficaz, nos mostrando o detalhes das capturas geradas, com isso conseguimos caracterizar o tráfego, identificar as portas utilizadas e com isso qual o serviço utilizado, por exemplo.
    \par No relatório mostramos como podemos verificar tais informações, filtra os dados da captura, mostrando sua sintaxe e complexidade na geração dos filtro.
    Claro, estamos longe de sermos especialistas na caracterização e análise, mas podemos afirmar que estamos caminhando na direção correta.

  \end{flushleft}

\end{document}
